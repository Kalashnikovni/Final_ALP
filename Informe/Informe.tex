\nonstopmode
\documentclass[a4paper,10pt]{article}
\usepackage[utf8]{inputenc}
\usepackage{textcomp}
\usepackage{tasks}
%\usepackage[spanish]{babel}
\usepackage{amsmath}
\usepackage{amssymb}
\usepackage[margin=1in]{geometry}
\usepackage{indentfirst}
\usepackage{graphicx}
\usepackage[nounderscore]{syntax}

\setlength{\grammarindent}{1.5cm}

\author{
    Marzorati, Denise \\
    \texttt{M-6219/7}
}

\date{
    2017
}

\title{
    \begin{center}
    \includegraphics[width=6cm, height=6cm]{logo-unr.png}
    \end{center}
    \Huge \textsc{{\bfseries T}rabajo {\bfseries P}ráctico {\bfseries F}inal} \\
    \Huge \textsc{Bin packing de polígonos simples} \\
    \large \textsc{Análisis de lenguajes de programación} \\
}

\begin{document}

\bigskip
\bigskip
\bigskip

\maketitle

\thispagestyle{empty}

\begin{center}
\large \bf Docentes de la materia
\end{center}

\begin{center}

Mauro Jaskelioff

Cecilia Mazino

Eugenia Simich

Juan Martín Rabasedas

\end{center}

\large
\newpage{}

\section*{Definiendo el problema}

\textit {Motivación}

    \smallskip
    Dada una plancha de madera de un ancho \textit{A} y largo \textit{L}, en los aserraderos tan solo se
    realizan cortes longitudinales o transversales de la misma para la venta al público. Por este motivo es
    necesario que el cliente que se acerca a realizar su compra minimice el "área rectangular" que ocuparán
    las figuras con las cuales luego trabajará. Al hablar de área rectangular se hace referencia al área del 
    rectángulo de menor área que contiene a todas las figuras dadas por el cliente.

\medskip
\textit {Consideraciones}
    \begin{itemize}
        \item No se trabajará con figuras arbitrarias, sino que simplificará el problema considerando tan solo
    polígonos simples; es decir, polígonos cuyos lados no adyacentes no se intersecan (ya que el caso de
    polígonos complejos puede reducirse a un problema de bin packing de polígonos simples). Por
    otro lado, la elección de polígonos como elementos de trabajo se basa en la posibilidad de chequear
    fácilmente si un par de ellos se encuentran solapados o no (es decir, si hay intersección entre sus lados).
        \item Si bien en el caso real se cuentan con varias planchas del mismo tamaño (con lo cual si las
    figuras no caben en una sola de ellas se utilizan las restantes), aquí se determina si se pueden acomodar
    todas las figuras dentro de una única plancha, devolviendo la correspondiente solución o bien un error.
        \item Se tiene en cuenta el material que quitará la herramienta de corte del cliente al cortar cada
    polígono, dado por el espesor y el tipo de la herramienta. A este valor se le denomina "kerf". 
    \end{itemize}

\section*{Instalación}

\smallskip
    Los módulos de Haskell que deben estar instalados son los siguientes:

\normalsize
\begin{tasks}(3)
    \task[+] AI.GeneticAlgorithm.Simple
    \task[+] Algorithms.Geometry
    \task[+] Control.DeepSeq
    \task[+] Control.Exception
    \task[+] Control.Lens
    \task[+] Control.Monad
    \task[+] Data.ByteString
    \task[+] Data.Char
    \task[+] Data.Ext
    \task[+] Data.Geometry
    \task[+] Data.List
    \task[+] Data.Maybe
    \task[+] Data.Set
    \task[+] Data.String
    \task[+] Data.Text
    \task[+] GHC.Float
    \task[+] Graphics.Gloss
    \task[+] System.Console.Readline
    \task[+] System.Directory
    \task[+] System.Environment
    \task[+] System.IO
    \task[+] System.Random
    \task[+] Text.Blaze
    \task[+] Text.PrettyPrint.HughesPJ
    \task[+] Text.XmlHtml
\end{tasks}

\large
    Para hacer uso del software tan solo es necesario compilar el archivo Main.hs con ghc, y ejecutar el
    ejecutable generado a partir de dicha compilación.  

\section*{Organización de los archivos}
\begin{itemize}
    \item \textit{Common.hs}: en este archivo se encuentran las definiciones de tipos de datos referidos a
    polígonos, rectángulos, máquina (que es un conjunto de polígonos y rectángulos junto con el kerf), y tipos
    de datos para los elementos pertinentes de archivos SVG.
    \item \textit{Draw.hs}: aquí se encuentran las funciones encargadas de crear el texto que conformará el
    archivo SVG que se devuelve como resultado.
    \item \textit{Eval.hs}: se evalúan las definiciones, obteniendo los polígonos resultantes de aplicar las
    transformaciones que correspondan, y chequeando que los mismos sean válidos (es decir, tengan al menos
    tres lados y sean polígonos simples). Además se le agrega el valor del kerf a cada polígono de manera tal
    que al ejecutar el algoritmo genético lo tenga en cuenta.
    \item \textit{GA.hs}: etapa del algoritmo genético, para revisar los detalles acerca del mismo ver [1].
    \item \textit{Main.hs}: desarrolla el intérprete del programa que acepta como entrada archivos cuya
    sintaxis describe polígonos, como así también archivos SVG que tengan definidos polígonos o rectángulos.
    Se utilizó como guía [2].
    \item \textit{Parse.hs}: parsea sentencias del DSL diseñado (de "máquinas").
    \item \textit{ParseSVGElements.hs}: parsea las componentes que son de interés de los elementos deseados de
    un archivo SVG.
    \item \textit{ParseSVG.hs}: utiliza bibliotecas ya desarrolladas para "pre parsear" un archivo SVG,
    quedándose tan solo con los elementos que son de interés del mismo, y devolviendo texto con la sintaxis
    que ParseSVGElements acepta como argumento de entrada.
    \item \textit{Polygon.hs}: contiene funciones utilizadas en la última etapa del proceso donde una vez que
    el algoritmo genético ha finalizado se vuelven a "introducir" los polígonos dentro de su respectivo
    contenedor.
    \item \textit{PrettyPrint.hs}
    \item \textit{Shrink.hs}: se encarga de la etapa de acercamiento de los polígonos.
\end{itemize}

\section*{DSL de "máquinas"}
\setlength{\leftskip}{0.4cm}
% *********************************************************************************************************** %
% *********************************************************************************************************** %
% *********************************************************************************************************** %
\subsection*{Sintaxis abstracta}
\large

\begin{grammar}
<floatExp> ::= <float>
    \alt $-_{1}$ <float>

<point> ::= <floatExp> $\sim$ <floatExp>

<polygon> ::= <point>
    \alt <point> , <point>

<container> ::= <float> $\times$ <float>

<defExp> ::= {\bfseries pol} <name> = <polygon>
    \alt {\bfseries con} <name> = <container>

<defs> ::= <empty>
    \alt <defExp> <defs>
    \alt {\bfseries pol} <name> = <polygon> {\bfseries copy} <natural> {\bfseries scale} <float> <defs>
    \alt {\bfseries con} <name> = <container> {\bfseries copy} <natural> {\bfseries scale} <float> <defs>

<machine> ::= {\bfseries kerf} <float> <defs>
\end{grammar}

\normalfont

% *********************************************************************************************************** %
% *********************************************************************************************************** %
% *********************************************************************************************************** %
\setlength{\leftskip}{0.4cm}
\subsection*{Sintaxis concreta}

\begin{grammar}
<digit> ::= '0' | '1' | ... | '9'

<natural> ::= <digit>
    \alt <digit> <natural>

<float> ::= <natural> '.' <natural>

<floatExp> ::= <float>
    \alt '-' <float>

<point> ::= <floatExp> '$\sim$' <floatExp>

<polygon> ::= <point>
    \alt <point> ',' <polygon>

<bPolygon> ::= '[' ']'
    \alt '[' <polygon> ']'

<container> ::= <float> '$\times$' <float>

<defExp> ::= {\bfseries pdef} <name> '=' <bPolygon>
    \alt {\bfseries 'con'} <name> '=' <container>

<defs> ::= <defExp> <defs>
    \alt {\bfseries pdef} <name> '=' <bPolygon> {\bfseries 'copy'} <natural> {\bfseries 'scale'} <float> <defs>
    \alt {\bfseries cdef} <name> '=' <container> {\bfseries 'copy'} <natural> {\bfseries 'scale'} <float> <defs>
    \alt

<machine> ::= {\bfseries 'kerf'} <float> <defs>
\end{grammar}

\normalfont

% *********************************************************************************************************** %
% *********************************************************************************************************** %
% *********************************************************************************************************** %
\section*{DSL de SVG}

    SVG cuenta con varios tipos de elementos que describen distintas figuras; los elementos que pueden
    resultar de interés para este trabajo son los siguientes: $<$rect$>$, $<$polygon$>$, $<$polyline$>$ y
    $<$path$>$. Se utilizó el módulo Text.XmlHtml para obtener los elementos que son de importancia para
    el objetivo planteado (por ejemplo: la rotación de una figura no es relevante), con lo cual luego dicho
    resultado debe ser parseado. A continuación se listan los campos relevantes de cada elemento, en el orden
    en que serán pasados al parser:

    \begin{itemize}
        \item $<$rect$>$: width, height, transform,  id. A este elemento solo se le permiten las
        transformaciones de translación, rotación o escalamiento (ya que transformaciones arbitrarias pueden
        dar como resultado otro tipo de cuadriláteros que no son rectángulos).
        \item $<$polygon$>$: points, transform,  id.
        \item $<$polyline$>$: points, transform,  id.
        \item $<$path$>$: d, transform,  id.
    \end{itemize}

    Para más detalles acerca de la sintaxis de dicho lenguaje ver las referencias.

\subsection*{Sintaxis abstracta}

\begin{grammar}
<floatExp> ::= <float>
    \alt $-_{1}$ <float>

<cPoint> ::= <floatExp> , <floatExp>

<point> ::= <floatExp> <floatExp>
    \alt <cPoint>

<rectList> ::= <empty>
    \alt {\bfseries translate} <floatExp> <rectList>
    \alt {\bfseries translate} <cPoint> <rectList>
    \alt {\bfseries rotate} <floatExp> <rectList>
    \alt {\bfseries scale} <floatExp> <rectList>

<sufixR> ::= <rectList> <name>

<rect> ::= <floatExp> <floatExp> <sufixR>

<transform> ::= <translate> <floatExp>
    \alt {\bfseries translate} <floatExp> <cPoint>
    \alt {\bfseries translate} <cPoint> <rectList>
    \alt {\bfseries rotate} <floatExp>
    \alt {\bfseries scale} <floatExp>
    \alt {\bfseries skewX} <floatExp>
    \alt {\bfseries skewY} <floatExp>
    \alt {\bfseries matrix} <floatExp> , <floatExp> , <floatExp> , <floatExp> , <floatExp> , <floatExp>

<transformList> ::= <empty>
    \alt <transform> <transformList>

<sufix> ::= <transformList> <name>

<pointList> ::= <empty>
    \alt <point> <pointList>

<polygon> ::= <pointList> <sufix>

<polyline> ::= {\bfseries poly} <pointList> <sufix>

<startPath> ::= {\bfseries M} <point> <pathLines>
    \alt {\bfseries m} <point> <pathLines>

<pathLines> ::= {\bfseries h} <floatExp> <pathLines>
    \alt {\bfseries H} <floatExp> <pathLines>
    \alt {\bfseries v} <floatExp> <pathLines>
    \alt {\bfseries V} <floatExp> <pathLines>
    \alt {\bfseries l} <floatExp> <pathLines>
    \alt {\bfseries L} <floatExp> <pathLines>
    \alt <point> <pathLines>
    \alt {\bfseries z}

<path> ::= <startPath> <sufix>
\end{grammar}

% *********************************************************************************************************** %
% *********************************************************************************************************** %
% *********************************************************************************************************** %
\subsection*{Sintaxis concreta}

\begin{grammar}
<digit> ::= '0' | '1' | ... | '9'

<natural> ::= <digit>
    \alt <digit> <natural>

<float> ::= <natural> '.' <natural>

<floatExp> ::= <float>
    \alt $-_{1}$ <float>

<char> ::= 'a' | 'b' | ... | 'z' | 'A' | 'B' | ... | 'Z'

<name> ::= <empty>
    \alt <char> <name>
    \alt <digit> <name>

<cPoint> ::= <floatExp> ',' <floatExp>

<point> ::= <floatExp> <floatExp>
    \alt <cPoint>

<rectList> ::= {\bfseries 'translate'} <floatExp> <rectList>
    \alt {\bfseries 'translate'} <cPoint> <rectList>
    \alt {\bfseries 'rotate'} <floatExp> <rectList>
    \alt {\bfseries 'scale'} <floatExp> <rectList>
    \alt

<sufixR> ::= <rectList> <name>

<rect> ::= <floatExp> <floatExp> <sufixR>

<transform> ::= <translate> <floatExp>
    \alt {\bfseries 'translate'} <floatExp> <cPoint>
    \alt {\bfseries 'translate'} <cPoint> <rectList>
    \alt {\bfseries 'rotate'} <floatExp>
    \alt {\bfseries 'scale'} <floatExp>
    \alt {\bfseries 'skewX'} <floatExp>
    \alt {\bfseries 'skewY'} <floatExp>
    \alt {\bfseries 'matrix'} <floatExp> ',' <floatExp> ',' <floatExp> ',' <floatExp> ',' <floatExp> ',' <floatExp>

<transformList> ::= <transform> <transformList>
    \alt

<sufix> ::= <transformList> <name>

<pointList> ::= <point> <pointList>
    \alt

<polygon> ::= <pointList> <sufix>

<polyline> ::= {\bfseries 'poly'} <pointList> <sufix>

<startPath> ::= {\bfseries 'M'} <point> <pathLines>
    \alt {\bfseries 'm'} <point> <pathLines>

<pathLines> ::= {\bfseries 'h'} <floatExp> <pathLines>
    \alt {\bfseries 'H'} <floatExp> <pathLines>
    \alt {\bfseries 'v'} <floatExp> <pathLines>
    \alt {\bfseries 'V'} <floatExp> <pathLines>
    \alt {\bfseries 'l'} <floatExp> <pathLines>
    \alt {\bfseries 'L'} <floatExp> <pathLines>
    \alt <point> <pathLines>
    \alt {\bfseries 'z'}

<path> ::= <startPath> <sufix>
\end{grammar}

% *********************************************************************************************************** %
% *********************************************************************************************************** %
% *********************************************************************************************************** %
\section*{Observaciones}

\begin{itemize}
    \item En varios de los módulos donde se deben realizar cálculos se utilizan cotas superiores o inferiores
    debido a la inexactitud de las operaciones (por ejemplo al considerar la intersección de dos rectas, si
    esta coincide con un punto dado).
\end{itemize}

% *********************************************************************************************************** %
% *********************************************************************************************************** %
% *********************************************************************************************************** %
\newpage{}
\begin{thebibliography}{9}

\bibitem{lamport94}
    Stefan Jakobs,
    \textit{On genetic algorithms for the packing of polygons},
    European journal of operational research,
    1996.

\bibitem{lamport95}
    Mauro Jaskelioff, Cecilia Manzino, Juan Manuel Rabasedas, Eugenia Simich
    \textit{Intérprete de lambda cálculo},
    FCEIA,
    2017.

\bibitem{lamport96}
    W3C, 
    \textit{Basic shapes},
    W3,
    16 August 2011,
    https://www.w3.org/TR/SVG/shapes,
    2017.
\end{thebibliography}

\end{document}
